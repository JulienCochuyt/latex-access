\documentclass[12pt,a4paper]{report}
\usepackage[pdftex]{hyperref}
\usepackage{tabularx}
\usepackage{amsmath}
\begin{document}
\title{The LaTeX-access manual}
\author{Alastair Irving <\href{mailto:alastair.irving@sjc.ox.ac.uk}{alastair.irving@sjc.ox.ac.uk}>\\
  Robin  Williams <\href{mailto:rmw205@exeter.ac.uk}{rmw205@exeter.ac.uk}>\\
  Daniel Dalton <\href{mailto:<daniel.dalton10@gmail.com}{daniel.dalton10@gmail.com}>\\
  Nathaniel Schmidt <\href{mailto:nathanieljsch@westnet.com.au}{nathanieljsch@westnet.com.au}>\\
  Stefan Moisei}
\maketitle
\tableofcontents

Latex-access -- provides a blind person with a more efficient means of
interacting with LaTeX in mathematical and scientific documents.

\chapter{Introduction}
\label{ch-introduction}

The latex-access project is designed to provide a realtime
translation of a line of LaTeX in to braille, concentrating
on the Nemeth code, whichh can be read on a refreshable braille
display. This will greatly improve the ease of use of
LaTeX to blind mathematicians and scientists. The project also
translates the current line into english speech which is
easier to listen to than LaTeX source.

Note that this project is largely aimed at people wishing
to read LaTeX using a refreshable braille display and/or
speech synthesisor[B, and people who will probably
want to edit LaTeX documents. For example, as a
university student,
I receive my worksheets in LaTeX format, and
produce my work using LaTeX. Using the latex\_access
package, I am able to
get a fairly good translation of the question
and then an on-the-fly translation of my work as
I produce it. If you are
not concerned with editing LaTeX documents
and simply want a braille translation of an
entire laTeX document, then this
project is not for you.

There is also a very low traffic mailing list, which is worth subscribing to if you have any
queries, problems, suggestions or ideas. All current developers are
subscribed to this list and are very willing to assist. To subscribe
send an email with the word ``subscribe'' in the subject to:
\href{mailto:latex-access-devel-request@lists.sourceforge.net}{latex-access-devel-request@lists.sourceforge.net}. To
post to the list send emails to \href{mailto:latex-access-devel@lists.sourceforge.net}{latex-access-devel@lists.sourceforge.net}

\section{Purpose}
\label{subchap-purpose}
It is widely thought that LaTeX is a good system for a blind
mathematician or scientist to use to create and read
scientific documents, as it is a linear code and so the user does
not have to perceive two-dimensional concepts, such
as fractions and column vectors.
By reading this linear code, a blind person can take in and
understand scientific documents in the same way that a
sighted person would do by studying a printed document.
It should be noted that normally, laTeX is just a source from which
documents are converted in to an
attractive-looking, typeset document that can be printed or viewed
on screen, often in a .pdf, .dvi or .ps format. For
various technical reasons, documents in such formats are
currently inaccessible with current screen-reading technology.
The best current solution therefore is not to concern
ourselves with documents in these formats, but rather to
read
and interpret the LaTeX source code itself.

\section{Reading a LaTeX document}
\label{subchap-reading-latex-document}

It is entirely possible to read a LaTeX document simply by reading
the LaTeX source itself. This however, is often a
time-consuming and pain-staking process, and it is often not
particularly nice to read. For example, the LaTeX source
for the quadratic formula is\\
\mbox{\$\$x=\textbackslash frac\{-b\textbackslash pm\textbackslash sqrt\{b\^2-4ac\}\}\{2a\}\$\$}\\
It is therefore the aim of the project to translate a
line of LaTeX in to a line of Niemeth braille code, which
can be
read using a refreshable braille display. The project
also aims to provide an audible translation of the
LaTeX source
which will be output through current screen-reading
technology.          

\section{Current features}
\label{subchap-current-features}


latex-access currently contains the following features.

\begin{itemize}
\item Translation of several mathematical expressions from LaTeX to
Niemeth braille. These include, but are not confined
to:
\begin{itemize}
\item Translation of fractions, both numerical and
algebraic.
\item Translation of trigonometric
functions and hyperbolic functions.
\item Translation of powers,
including square roots.
\item Translation of
expressions used in calculus, including partial derivatives.
\item Translation of two component and three component column vectors,
  not in to Niemeth braille format but in to a row vector so that they
  can be read on a single line braille display.
\item Translation of several mathematical symbols, such as the Greek letters.
\item Many commands used to create a visually attractive document are
  either translated or ignored, often it is not necessary to see some
  formatting commands.
\end{itemize}

\item Translation of several of the above to audible speech.
\item A matrix browser feature to enable easier reading of larger
  matrices in LaTeX, see the description below.
\item Support for custom defined LaTeX commands.
\end{itemize}

\chapter{Obtaining the source}
\label{ch-obtaining-the-source}

The package is hosted by svn at,
\url{https://latex-access.svn.sourceforge.net/svnroot/latex-access}.

This link will take you to a web interface of the svn tree, but you'll
probably want to checkout the code so you can install it. If you run
windows see section~\ref{subchap-windows}, and if you run linux
see section~\ref{subchap-linux}.

\section{Linux}
\label{subchap-linux}

Under Linux, the standard subversion command line client works
well. This can usually be installed on debian based distros by running\\
\mbox{apt-get install subversion}\\
\# (as root)

Then type\\
\mbox{svn co https://latex-access.svn.sourceforge.net/svnroot/latex-access latex-access}

This will check the package out into the directory latex-access.

In future feel free to run ``svn up'', to pull the latest updates from
the server. (from within the latex-access directory).

\section{Windows}
\label{subchap-windows}

On windows the ``Tortoise SVN'' client works well.\\
\url{http://tortoisesvn.tigris.org/}

Once you have checked out the sourcecode from\\
\url{https://latex-access.svn.sourceforge.net/svnroot/latex-access}\\
continue with the installation
process.

Note, you should periodically pull the latest updates from the server to
get the latest and greatest features of latex-access.

\chapter{Installation}
\label{ch-instalation}
\chapter{General usage}
\label{ch-general-usage}
\chapter{More advance usage}
\label{ch-more-advanced-usage}
\chapter{Extra components}
\label{ch-extra-components}
\chapter{For developers}
\label{ch-for-developers}
\appendix

\chapter{Key bindings}
\label{APA}
\chapter{Target specific notes}
\label{APB}
\chapter{Future features}
\label{APC}

\end{document}